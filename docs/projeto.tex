%% abtex2-modelo-projeto-pesquisa.tex, v<VERSION> laurocesar
%% Copyright 2012-2015 by abnTeX2 group at http://www.abntex.net.br/
%%
%% This work may be distributed and/or modified under the
%% conditions of the LaTeX Project Public License, either version 1.3
%% of this license or (at your option) any later version.
%% The latest version of this license is in
%% http://www.latex-project.org/lppl.txt
%% and version 1.3 or later is part of all distributions of LaTeX
%% version 2005/12/01 or later.
%%
%% This work has the LPPL maintenance status `maintained'.
%%
%% The Current Maintainer of this work is the abnTeX2 team, led
%% by Lauro César Araujo. Further information are available on
%% http://www.abntex.net.br/
%%
%% This work consists of the files abntex2-modelo-projeto-pesquisa.tex
%% and abntex2-modelo-references.bib
%%

% ------------------------------------------------------------------------
% ------------------------------------------------------------------------
% abnTeX2: Modelo de Projeto de pesquisa em conformidade com
% ABNT NBR 15287:2011 Informação e documentação - Projeto de pesquisa -
% Apresentação
% ------------------------------------------------------------------------
% ------------------------------------------------------------------------

\documentclass[
% -- opções da classe memoir --
12pt,				% tamanho da fonte
openright,			% capítulos começam em pág ímpar (insere página vazia caso preciso)
oneside,			% para impressão em recto e verso. Oposto a oneside
a4paper,			% tamanho do papel.
% -- opções da classe abntex2 --
% chapter=TITLE,		% títulos de capítulos convertidos em letras maiúsculas
% section=TITLE,		% títulos de seções convertidos em letras maiúsculas
% subsection=TITLE,	% títulos de subseções convertidos em letras maiúsculas
% subsubsection=TITLE,% títulos de subsubseções convertidos em letras maiúsculas
% -- opções do pacote babel --
english,			% idioma adicional para hifenização
brazil,				% o último idioma é o principal do documento
]{abntex2}


% ---
% PACOTES
% ---

% ---
% Pacotes fundamentais
% ---
\usepackage{lmodern}			% Usa a fonte Latin Modern
\usepackage[T1]{fontenc}		% Selecao de codigos de fonte.
\usepackage[utf8]{inputenc}		% Codificacao do documento (conversão automática dos acentos)
\usepackage{indentfirst}		% Indenta o primeiro parágrafo de cada seção.
\usepackage{color}				% Controle das cores
\usepackage{graphicx}			% Inclusão de gráficos
\usepackage{microtype} 			% para melhorias de justificação
\usepackage{amsmath,amssymb,amstext}
\usepackage{setspace}
\usepackage{float}

% ---
% Pacotes e definições adcionais, para adequações especificas
\usepackage{tikz}
\usepackage{pdflscape}			% para ambiente landscape
\usepackage{pgfgantt}			% cronograma estilo gráfico de gantt
\usetikzlibrary{backgrounds}
\definecolor{done}{RGB}{120, 180, 120}
\definecolor{do}{RGB}{180, 120, 120}

% ---

% ---
% Pacotes adicionais, usados apenas no âmbito do Modelo Canônico do abnteX2
% ---
\usepackage{lipsum}				% para geração de dummy text
% ---

% JSON
\usepackage{listings}
% 2 columns
\usepackage{parcolumns}
\usepackage{adjustbox}

% ---
% Pacotes de citações
% ---
\usepackage[brazilian,hyperpageref]{backref}            % Paginas com as citações
\usepackage[alf]{abntex2cite}				% Citações padrão ABNT
% \usepackage{abntex2cite}

% ---
% CONFIGURAÇÕES DE PACOTES
% ---

% ---
% Configurações do pacote backref
% Usado sem a opção hyperpageref de backref
\renewcommand{\backrefpagesname}{Citado na(s) página(s):~}
% Texto padrão antes do número das páginas
\renewcommand{\backref}{}
% Define os textos da citação
\renewcommand*{\backrefalt}[4]{
  \ifcase #1 %
  Nenhuma citação no texto.%
  \or
  Citado na página #2.%
  \else
  Citado #1 vezes nas páginas #2.%
  \fi}%
% ---

% ---
% Informações de dados para CAPA e FOLHA DE ROSTO
% ---
\titulo{Dota2}
\vspace{2cm}
\autor{Andryas Waurzenczak}
\local{Curitiba}
\data{2019}
\instituicao{Universidade Federal do Paraná
  \par
  Setor de Ciências Exatas
  \par
  Departamento de Estatística
}
\orientador[Orientador:]{Prof. Dr. Walmes Marques Zeviani}
\tipotrabalho{Projeto de Pesquisa}
% O preambulo deve conter o tipo do trabalho, o objetivo,
% o nome da instituição e a área de concentração
\preambulo{Projeto de Pesquisa apresentado à disciplina Laboratório A
  do Curso de Graduação em Estatística da Universidade Federal do Paraná,
  como requisito para elaboração do Trabalho de Conclusão de Curso}
% ---

% ---
% Configurações de aparência do PDF final

% alterando o aspecto da cor azul
\definecolor{blue}{RGB}{41,5,195}

% informações do PDF
\makeatletter
\hypersetup{
  % pagebackref=true,
  pdftitle={\@title},
  pdfauthor={\@author},
  pdfsubject={\imprimirpreambulo},
  pdfcreator={LaTeX with abnTeX2},
  pdfkeywords={abnt}{latex}{abntex}{abntex2}{projeto de pesquisa},
  colorlinks=true,	% false: boxed links; true: colored links
  linkcolor=blue,     % color of internal links
  citecolor=blue, % color of links to bibliography
  filecolor=magenta, % color of file links
  urlcolor=blue,
  bookmarksdepth=4
}
\addto\captionsbrazil{
  \renewcommand{\bibname}{REFER\^ENCIAS}
}
\makeatother
% ---

% ---
% Espaçamentos entre linhas e parágrafos
% ---

% O tamanho do parágrafo é dado por:
\setlength{\parindent}{1.3cm}

% Controle do espaçamento entre um parágrafo e outro:
\setlength{\parskip}{0.2cm}  % tente também \onelineskip


% ---
% compila o indice
% ---
\makeindex
% ---

% ----
% Início do documento
% ----
\begin{document}

% Seleciona o idioma do documento (conforme pacotes do babel)
% \selectlanguage{english}
\selectlanguage{brazil}

% Retira espaço extra obsoleto entre as frases.
% \frenchspacing

% ----------------------------------------------------------
% ELEMENTOS PRÉ-TEXTUAIS
% ----------------------------------------------------------
% \pretextual

% ---
% Capa
% ---
\tikz[remember picture,overlay] \node[opacity=0.9,inner sep=0pt] at
(current page.center){
  \includegraphics[width=\paperwidth,
  height=\paperheight]{image/ufpr_bg}};

\begin{center}
  {\Large \textsf{Universidade Federal do Paraná}} \\
\end{center}
\imprimircapa

% ---

% ---
% Folha de rosto
% ---
\imprimirfolhaderosto
% ---

% ---
% inserir o sumario
% ---
\pdfbookmark[0]{\contentsname}{toc}
\tableofcontents*
\cleardoublepage
% ---


% ----------------------------------------------------------
% ELEMENTOS TEXTUAIS
% ----------------------------------------------------------
\textual

% ----------------------------------------------------------
% Introdução
% ----------------------------------------------------------
\chapter{Introdução}
% \label{cha:introducao}

Dos astrágalos da antiga Grécia aos jogos de console e computador do século XXI.
Ao longo da trágetoria humana os jogos tem-se mostrado uma atividade essêncial
para o desenvolvimento tecnológico e cientifico. Uma das primerias teorias sobre
a lei das probabilidade, que é um dos alicerces da ciência estatística, em suas
primeiras formulações por Girolamo Cardano, teve como motivação, segundo
\cite{bernstein1997desafio}, o desenvolvimento de uma teoria dos jogos, e não
uma teoria de probabilidades.


Ainda no século XXI os jogos continuam a fomentar o desenvolvimento de novas
tecnologias, hoje ouvesse muito sobre Inteligência Artificial (AI) e Máquina de
Aprendizado (ML) que são ramificações da estatística que ganharam vida
própria. No entanto, o progresso e o sucesso dessas áreas está ligada, de certa
forma, aos jogos, pois se no momento em que se consegue fazer um computador
jogar um jogo tão complexo como Dota2, por exemplo, onde deve-se tomar diversar
decisões simultãneas instante à instante, qual seria a dificuldade de
externelizar esse conhecimento para o mundo real? Pode-se dizer que jogos de
computadores são verdadeiros laborátorios para o desenvolvimento e o progresso da
ciência como um todo \cite{silva2017moba}.

Não obstante, os jogos movimentam a economia, fatos sobre isso podem ser
observados em estatísticas que revistas especializadas sobre mercados de jogos
disponibilizam, segundo a \cite{newzoo2018global}, por exemplo, o mercado de
jogos no ano de 2018 teve um lucro de aproximadamente de 134.9 bilhões de
doláres, onde 63.2 bilhões para jogos de mobile, 38.3 bilhões para jogos de
console e 33.4 bilhões para jogos de computador. E além disso abriu espaço 
para diversos sistemas de apostas que fornecem serviços nos mais
variados jogos eletrônicos, alguns exemplos são
gg.bet\footnote{https://gg.bet/pt/betting},
pinnacle\footnote{https://www.pinnacle.com/pt/}.

Este trabalho tem como objetivo contribuir com a ciência dos esportes
eletrônicos, no genêro MOBA (Multiplayer Online Battle Arena) no jogo
DotA2\footnote{www.dota2.com}. A escolha desse jogo se deve a alguns fatores,
primeiramente pela preferência do autor, segundo por ser o eSport mais bem pago
de todos os tempos e por último porque Elon Musk, um dos maiores empreendedores
e visionarios do mundo, investiu em um projeto conhecido como
openAI\footnote{https://openai.com/} que desenvolveu algoritmos inteligentes
capazes de jogar o jogo pela primeira vez no The International de 2017 o que
chocou a comunidade e fez o jogo ganhar mais espaço e atrair mais cientistas das
mais diversas áreas.


% ----------------------------------------------------------
% Objetivos
% ----------------------------------------------------------
\chapter{Objetivos}
% \label{cha:objetivos}

\section{Objetivos Gerais}
% \label{sec:objetivosgerais}

Avaliar se inclusão de variáveis relacionadas ao jogador tem impacto
significativo em prever o desfecho da partida após a seleção dos heróis no
jogo Dota2.

\section{Objetivos Específicos}
% \label{sec:objetivosespecificos}

\begin{itemize}
\item Apresentar a temática eSports

\item Apresentar a forma de coleta dos dados via API

\item Apresentar e Implementar o processo de tabulação dos dados para fins de
modelagem

\item Ajustar, validar e comparar diferentes modelos com e sem a informação dos
jogadores, atráves de métricas de desempenho

\end{itemize}

% ----------------------------------------------------------
  % Materiais e Métodos
% ----------------------------------------------------------
\chapter{Materiais e Métodos}
\label{cha:materiaisemetodos}

\section{Materiais}
\label{sec:materiais}

\subsection{Sobre os Dados}

Os dados são coletados atráves de uma \emph{Application programming interface}
(API) da plataforma de jogos Steam\footnote{https://store.steampowered.com/about/}.
Este serviço é disponivel somente para usúarios que tenham comprado algum jogo
da plataforma, pois, somente após a compra é disponibilizado uma chave de
acesso. Cada chave pode fazer uma quantidade limitada de requisições em um curto
espaço de tempo, para este trabalho cujo o objetivo é avaliar variáveis
relacionadas ao jogador presente em cada partida tem-se um primeiro problema na
aquisição dos dados, sendo que para cada partida coletada é necessário buscar o
histórico de cada jogador presente na partida, isto faz com que a quantidade de
requisições cresca tendo assim problemas na coleta dos dados. Na seção \ref{sec:coletadosdados}
 é apresentado o fluxo de execução programada e dado
maiores detalhes para contornar este problema.

\subsection{Recursos Computacionais}

Para as análises e desenvolvimento do trabalho será utilizado o software R \cite{rcoreteam}.
Para o armazenamento e a tabulação das informações será
utilizado dois banco de dados, o primeiro é o
MongoDB\footnote{https://www.mongodb.com/}, que é um banco de dados
destruturados e o segundo é o MySQL\footnote{https://www.mysql.com/} que é
um banco de dados relacional para a tabulação dos dados.


\subsection{Coleta dos Dados}
\label{sec:coletadosdados}

O processo de coleta dos dados é feito atráves de programas em R que ficam
coletando dados 24 horas por dias fazendo requisições na API da plataforma
Steam para a aquisião de novos IDs de partidas, preparando as informações da
partida e dos jogadores que estão presentes nela e armazenando no banco de dados
MongoDB.


O fluxo abaixo trás mais detalhes deste processo. As caixas em cinzas é a ordem
da execução/fase de cada programa.

\begin{figure}[H]
  \begin{center}
    \includegraphics[width=14cm,height=10cm]{image/collect.png}
    \caption{Fluxo do processo de coleta dos dados}
  \end{center}
\end{figure}

\begin{enumerate}
\item O programa \textbf{get\_id.R} faz uma requisição na API e retorna com as
  partidas que estão acontencendo no momento da requisição, os IDs dessas
  partidas são processados e armazenados, e repete-se o processo de tempos em
  tempos (120 segundos). É necessário o armazenamento dos IDs para uma coleta
  posterior das informações da partida após seu termino, pois é possível
  coletar as informações da partida se estas ainda não terminaram. Deve-se
  deixa-lo rodando em uma máquina 24 horas por dia com conexão estável com a
  internet. Para este trabalho um computador da
  Amazon\footnote{https://aws.amazon.com/pt/} está sendo utilizado.

\item O programa \textbf{split.R} divide o arquivo  gerado pelo programa
  \textbf{get\_id.R}, que é um vetor com \textbf{n} IDs coletados que são
  separados em \textbf{j} partes, para cada parte uma chave de acesso é
  necessária, para processar mais rapidamente as informaçõesda partida.

\item O programa \textbf{collect.R} lê cada um dos arquivos gerados pelo
  programa \textbf{split.R} e cria uma instância do R, para cada um, que irá
  coletas as informações da partida e armazena-las no MongoDB.

\item Após coletar as informações da partida o arquivo \textbf{split.R}
  conecta-se ao banco de dados, seleciona todas as partidas coletadas e pega os
  IDs de todos os jogadores que compoem a partida. Após isso divide em
  \textbf{j} arquivos .RData, cada qual com informações do ID da partida e o ID
  do jogador que fez parte dela.

\item Por último o programa \textbf{collect.R} processa cada arquivo com o ID da
  partida e o ID do jogador varrendo o histórico do jogador coletando as suas
  últimas 10 partidas e armazenando-as no banco de dados MongoDB.
  
\end{enumerate}

Como exemplo, tem-se abaixo as informações de uma única partida coletada. O
campo \emph{players} está oculto na primeira representação pois este campo
contém as informações de cada jogador na partida, e é demasiadamente longo.
Para ilustrar foi selecionado apenas o primeiro jogador que está na próxima
página.


\lstset{
  % basicstyle=\tiny
  string=[s]{"}{"},
  stringstyle=\color{blue},
  comment=[l]{:},
  commentstyle=\color{black},
  basicstyle=\tiny
}

\begin{center}
\begin{minipage}{.5\textwidth}
\begin{lstlisting}[caption=match]{Name}
[
  {
    "players": ...,
    "radiant_win": false,
    "duration": 1869,
    "pre_game_duration": 90,
    "start_time": 1549408252,
    "match_id": 4394571998,
    "match_seq_num": 3799941136,
    "tower_status_radiant": 390,
    "tower_status_dire": 1974,
    "barracks_status_radiant": 51,
    "barracks_status_dire": 63,
    "cluster": 184,
    "first_blood_time": 97,
    "lobby_type": 7,
    "human_players": 10,
    "leagueid": 0,
    "positive_votes": 0,
    "negative_votes": 0,
    "game_mode": 22,
    "flags": 1,
    "engine": 1,
    "radiant_score": 19,
    "dire_score": 36,
    "picks_bans": [
      {
        "is_pick": false,
        "hero_id": 8,
        "team": 0,
        "order": 0
      },
      {
        "is_pick": false,
        "hero_id": 44,
        "team": 0,
        "order": 1
      }
    ]
  }
]
\end{lstlisting}
\end{minipage}
\end{center}

\begin{adjustbox}{width=1.3\textwidth, height=12cm, keepaspectratio}
\begin{minipage}{.6\textwidth}
\begin{lstlisting}[caption="players", captionpos=b]{Name}
  [{
    "players": [
    {
      "account_id": 292190898,
      "player_slot": 0,
      "hero_id": 54,
      "item_0": 50,
      "item_1": 151,
      "item_2": 11,
      "item_3": 252,
      "item_4": 112,
      "item_5": 36,
      "backpack_0": 0,
      "backpack_1": 0,
      "backpack_2": 0,
      "kills": 2,
      "deaths": 5,
      "assists": 6,
      "leaver_status": 0,
      "last_hits": 201,
      "denies": 28,
      "gold_per_min": 440,
      "xp_per_min": 462,
      "level": 18,
      "hero_damage": 16096,
      "tower_damage": 2198,
      "hero_healing": 1040,
      "gold": 524,
      "gold_spent": 13110,
      "scaled_hero_damage": 9318,
      "scaled_tower_damage": 1216,
      "scaled_hero_healing": 603,
      "ability_upgrades": 
      {
        "ability": 5250,
        "time": 284,
        "level": 1
      },
      {
        "ability": 5251,
        "time": 395,
        "level": 2
      },
      {
        "ability": 5250,
        "time": 490,
        "level": 3
      },
      {
        "ability": 5249,
        "time": 566,
        "level": 4
      },
      {
        "ability": 5250,
        "time": 688,
        "level": 5
      },
      ...
    },
    ...
    ]
\end{lstlisting}
\end{minipage}\hfill
\begin{minipage}{.6\textwidth}
\begin{lstlisting}[caption="players.ability\_upgrades", captionpos=b]{Name}
  [{
    ...,
    {
      "ability": 5252,
      "time": 769,
      "level": 6
    },
    {
      "ability": 5249,
      "time": 846,
      "level": 7
    },
    {
      "ability": 5250,
      "time": 975,
      "level": 8
    },
    {
      "ability": 5251,
      "time": 1067,
      "level": 9
    },
    {
      "ability": 5906,
      "time": 1158,
      "level": 10
    },
    {
      "ability": 5249,
      "time": 1202,
      "level": 11
    },
    {
      "ability": 5253,
      "time": 1333,
      "level": 12
    },
    {
      "ability": 5249,
      "time": 1438,
      "level": 13
    },
    {
      "ability": 5251,
      "time": 1515,
      "level": 14
    },
    {
      "ability": 5939,
      "time": 1718,
      "level": 15
    },
    {
      "ability": 5251,
      "time": 1860,
      "level": 16
    },
    {
      "ability": 5252,
      "time": 2169,
      "level": 17
    }
    ]
  }
\end{lstlisting}
\end{minipage}
\end{adjustbox}

\section{Métodos}
\label{sec:metodos}

Para este trabalho se fará uso tanto de Modelos Lineares Generalizados \cite{nelder1972generalized}, em
específico o modelo Logito, como modelos Bayesianos e algoritmos de Machine
Learning. Parte essencial no processo de modelagem se fará no treinamento e na
validação dos modelos ajustados e na criação de características, e para isso
diversas técnicas poderão ser utilizadas como validação cruzada, holdount, curva
ROC etc...



% ----------------------------------------------------------
 % Cronograma
% ----------------------------------------------------------
%   \begin{landscape}
% \chapter{Cronograma de Atividades}
% \label{cha:cronograma}
%
% % Pretende-se cumprir o cronograma de atividades abaixo, estratificado
% % pelas semanas dos meses que compreendem a execução do trabalho.
% % \vspace{-0.2cm}
% \begin{center}
% \begin{SingleSpacing}
% \hspace{-1.5cm}
% % \resizebox{1\textwidth}{0.7\textheight}{
%   \begin{tikzpicture}[thick, scale=0.94, every node/.style={scale=0.94}]
%   \begin{ganttchart}[
%     canvas/.append style={fill=none},
%     y unit title=0.6cm, % Size da indicação do tempo
%     y unit chart=0.8cm, % Size do altura das colunas
%     x unit= 10mm, % largura das celulas
%     vgrid={*1{black!50, dotted}}, % grid cinza vertical
%     hgrid={*1{black!50, dotted}}, % grid cinza horizontal
%     title height=0.8, % Size dos dias
%     bar/.style={fill=done},
%     bar incomplete/.append style={fill=do},
%     bar label node/.append style={align=right},
%     bar label font=\scriptsize\color{black!65},
%     group label font=\bfseries\scriptsize\color{black},
%     group left peak width=0.2,
%     group right peak width=0.2,
%     group left peak height=0.15,
%     group right peak height=0.15,
%     bar height=0.6, % size das barras de tarefas
%     bar left shift=.2, bar right shift=-.2,
%     bar top shift=.2, bar height=.4,
%     link/.style={-to, line width=0.7pt, black!50},
%     link type=dr,
%     today=23,
%     today offset=0.8,
%     today label=Entrega dia 18/07,
%     today label font=\bfseries\scriptsize,
%     today rule/.style={draw=black, thick, dashed},
%     progress label text ={\pgfmathprintnumber[precision=0,
%                                               verbatim]{#1}\% realizado},
%                                                 ]{1}{24} %
%   \gantttitle{Fevereiro\footnote{ss}}{4}
%   \gantttitle{Março}{4}
%   \gantttitle{Abril}{4}
%   \gantttitle{Maio}{4}
%   \gantttitle{Junho}{4}
%   \gantttitle{Julho}{4} \\
%   \gantttitlelist{1,...,24}{1} \\
%
%   \ganttgroup[group label node/.append style={align=right},
%               progress=75]{Projeto \ganttalignnewline de Pesquisa}{1}{3} \\
%   \ganttbar[progress=100, progress label text=]{Entrega à
%     \ganttalignnewline banca}{1}{2} \\
%
%   \ganttgroup[group label node/.append style={align=right},
%               progress=20]{Elaboração \ganttalignnewline da Pesquisa}{2}{20} \\
%   \ganttbar[progress=40]{Revisão de \ganttalignnewline
%     literatura}{2}{8} \\
%   \ganttbar[progress=10]{Implementação
%     \ganttalignnewline Computacional}{7}{14} \\
%   \ganttbar[progress=0, progress label text= ]{Análise dos
%     \ganttalignnewline dados simulados}{11}{12} \\
%   \ganttbar[progress=0, progress label text= ]{Análise de
%     \ganttalignnewline dados reais}{11}{14} \\
%   \ganttbar[progress=0, progress label text= ]{Discussão
%     \ganttalignnewline dos resultados}{13}{16} \\
%   \ganttbar[progress=0, progress label text= ]{Redação
%     \ganttalignnewline da pesquisa}{15}{20} \\
%
%   \ganttgroup[group label node/.append style={align=right},
%               progress=0, progress label text=]{Defesa \ganttalignnewline do
%                 Trabalho}{20}{21} \\
%   \ganttbar[progress=0, progress label text= ]{Elaboração da
%     \ganttalignnewline apresentação}{20}{20} \\
%
%   \ganttgroup[group label node/.append style={align=right},
%               progress=0, progress label text=]{Revisão Final}{21}{24} \\
%   \ganttbar[progress=0, progress label text= ]{Incorporação
%     \ganttalignnewline das sugestões}{21}{22} \\
%   \ganttbar[progress=0, progress label text= ]{Redação da
%     \ganttalignnewline versão final}{22}{23} \\
%   \ganttlink{elem1}{elem2}
%   \ganttlink{elem3}{elem4}
%   \ganttlink{elem4}{elem5}
%   \ganttlink{elem4}{elem6}
%   \begin{scope}[on background layer]
%   \ganttlink{elem5}{elem7}
%   \end{scope}
%   \ganttlink{elem6}{elem7}
%   \ganttlink{elem3}{elem7}
%   \ganttlink{elem3}{elem8}
%   \ganttlink{elem7}{elem8}
%   \ganttlink{elem8}{elem10}
%   \ganttlink{elem10}{elem12}
%   \ganttlink{elem10}{elem13}
%   \ganttlink{elem12}{elem13}
%   \end{ganttchart}
%   \end{tikzpicture}
%   \end{SingleSpacing}
%   \end{center}
%
%   \end{landscape}
%   ---
\phantompart


% ----------------------------------------------------------
% ELEMENTOS PÓS-TEXTUAIS
% ----------------------------------------------------------
\postextual

% ----------------------------------------------------------
% Referências bibliográficas
% ----------------------------------------------------------
\bibliography{plain}


\end{document}
